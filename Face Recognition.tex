\documentclass{SBCbookchapter}
\usepackage[utf8]{inputenc}
\usepackage[T1]{fontenc}
\usepackage[brazil,english]{babel}
\usepackage{graphicx}
\title{Face Recognition}
\begin{document}
\maketitle

%\addcontentsline{toc}{chapter}{Statistics}

%\emph{Find a statistical method which uses only Japan Meteorological Agency (JMA) best track typhoon data during or before year $n$ to hind-cast the entire best track of all typhoons in years $n+1$.   Given only the latter typhoons' initial three best track points (i.e. at $t=0,6,$ and $12$ hours), the maximum position error at each hind-casted track point must be less than 500 km. The method must work for the three most recent years of published best track data.}

\vspace{.3in}
\begin{minipage}[r]{4.25in}{\small
  Child trafficking is a heinous crime which should be combatted by all means including advanced technology.  In the case of trans-national
  trafficking, border control may utilize face recognitio n technology to aid in the rescue of trafficked children.

\newpage

\section{Introduction}


\begin{figure}[h]
\hspace{1.2in} \includegraphics[width=4in, height=2.25in]{HaiyanTrackDefault.png}
 \caption{Default model simulation of Haiyan (2013) which exhibited a 1500 km track error at the 120 hr mark. See Table \ref{typ2}. }
 \centering
 \label{haiyanbasic}
\end{figure}


    \begin{table}[h]
\centering
\tiny
\begin{tabular}{|l|l|l|l|l|l|l|l|l|}
\multicolumn{9}{c}{$p$-NORM MODEL PARAMETERS }\\\hline
&$\Omega_0$ & $\Omega_1$ & $\Omega_2$ & $\Omega_3$ & $\Omega_4$ & $p$ &\multicolumn{2}{l|}{$\gamma$}\\\hline
Latitude &  .5 & 25 & .25 & 59.5 & 0 & 1.3 & \multicolumn{2}{l|}{.05} \\
Longitude &  0 & 0 & 0 & 0 & 0 & 0 & \multicolumn{2}{l|}{.35} \\
Intensity &  .01 & 1 & 10 & 5 & 0 & 1.2 & \multicolumn{2}{l|}{.2} \\
Pressure & .4 &1000&.65&20&5&1.1&\multicolumn{2}{l|}{.6} \\ \hline \hline
\multicolumn{9}{|c|}{POSITION ERRORS }\\\hline
  Hours in advance    & 24-hr & 48-hr &72-hr &96-hr & 120-hr & 144-hr & 168-hr& 192-hr\\\hline
  Default position error  (km)  &297.6  &477.3 & 728.0&1021.9 & 1498.8 &1874.3  & 2105.0 &  1975.6 \\\hline
  $p$-norm position error  (km)   &  125.1  &  97.2  & 46.6  & 65.5  & 167.3& 266.7  &  445.1&   634.6   \\\hline
  Relative error reduction & $R_1=.58 $  &$R_2=.8 $ &$R_3=.94 $ &$R_4=.94 $ &$R_5=.89 $ &$R_6=.86 $  &$R_7=.79 $ &  $R_{7.75} =.68 $   \\ \hline
 \end{tabular}
  \caption{Haiyan (international ID 1330) $p$-norm model simulation. DUIR: JMA grade 5 Philippine landfall (2009 up to Haiyan) }
  \label{typ2}
\end{table}

  \begin{table}[h]
\centering
\tiny
   \begin{tabular}{|l|l|l|l|l|l|l|l|}
\multicolumn{8}{c}{$p$-NORM MODEL PARAMETERS }\\\hline
$p$-norm model parameters&$\Omega_0$ & $\Omega_1$ & $\Omega_2$ & $\Omega_3$ & $\Omega_4$ & $p$ &$\gamma$\\\hline
Latitude &  5 & 9.5 & 20 & .2 & 1 & 1.73 & .26 \\
Longitude &  9 & 25 & 15 & .3 & 0 & .5 & .95 \\
Intensity &  20 & 1 & 10 & 5 & 0 & 1 & .5 \\
Pressure & 1 &10&15&15&12.5&1.1&.85 \\ \hline \hline
\multicolumn{8}{|c|}{POSITION ERRORS }\\\hline
  Hours in advance   & 24-hr & 48-hr &72-hr &96-hr & 120-hr &   144-hr& 168-hr\\\hline
 Default position error  (km)&  15.3  & 140.4    & 31.6  & 198.2   & 203.1 &  341.0 &747.1 \\\hline
 $p$-norm position error  (km)&  37.8  &  150.0  & 162.9  & 416.2  &  390.1& 151.1&159.1 \\\hline
 Relative error reduction & $R_1= -1.47 $  &$R_2=-.07 $ &$R_3=-4.16 $ &$R_4=-1.1 $ &$R_5=-.92 $ &$R_6=.56 $  &$R_7=.79 $ \\ \hline \hline
   Hours in advance   &  192-hr &216-hr & 240-hr &  264-hr & 288-hr &  312-hr & 336-hr \\\hline
 Default position error  (km)    & 1257.1  & 1892.1   & 2722.0 &3343.9  & 4140.6    & 4659.2  & 5088.5     \\\hline
 $p$-norm position error  (km)  & 348.3  & 647.7  &  1096.9&   1284.4  &  1548.0  & 1473.7  & 1126.3       \\\hline
 Relative error reduction & $R_8=.72 $ &$R_9=.66 $ &$R_{10}=.62 $   & $R_{11}=.60 $  &$R_{12}=.62 $ &$R_{13}=.68 $ &$R_{14}=.78$ \\ \hline \hline
 \end{tabular}
  \caption{Guchol (international ID 1204) $p$-norm model simulation. DUIR:  JMA best track grade 5 typhoon data (2009-)}
  \label{typ3}
 \end{table}
\begin{figure}[h]
\centering
  \includegraphics [width=2.5in, height=2.25in]{GucholTrackDefault.eps}
  \includegraphics[width=2.5in, height=2.25in]{GucholTrackp.eps}
     \caption{Missing curvature in the default model of Guchol (2012) is corrected by the $p$-norm model (solid markers=actual, dashed=simulated; parameters given in Table 3.)}
  \label{gucholfig}
\end{figure}



\section{Computer Project}
Write a program which can create SBTs for the default and $p$-norm models such as shown in Figure \ref{gucholfig}.





\vspace{1in}

\emph{Acknowledgement:} Guidance from Dr. Takemasa Miyoshi and the Journal of the Meteorological Society of Japan's editorial staff is gratefully acknowledged. This project was supported by Wheaton College's Summer Research Programs.
.
\vspace{.25in}

\emph{Project Team:} Korey Clement, Daniela Cuba, Michael Kietzman, Peyton Finley, Jacob Clement, Danilo Diedrichs, Roland Hesse, Spencer Hills, Kaile Phelps, Jenny Ruda, Erica Swain, Kei Takazawa, Emily Wilson

\newpage
 \begin{thebibliography}{9}

 \bibitem{Arakawa}  {\sc  ARAKAWA, H.}, 1964. Statistical method to forecast the movement and the central pressure of typhoons in the western north Pacific. \emph{J. Appl. Meteorol.} {\bf 3}, 524-528.

     \bibitem{Cap}   {\sc CAP, F.}, 200.: {\em Tsunamis and Hurricanes: A Mathematical Approach.} Springer-Verlag, 201 pp.

\bibitem{Chan} {\sc CHAN, J.C.L.}, 2005: The physics of tropical cyclone motion. \emph{Annu. Rev. Fluid Mech.} {\bf 37}, 99-128.
\bibitem{Hig}  {\sc HIGAKI, M.}, {\sc KYOUDA, M.} and {\sc YAMAGUCH, H.} 2015. Upgrade of JMA's Typhoon Ensemble Prediction System. http://www.wcrp-climate.org/ WGNE/ Blue Book/2014/individual-articles/06\_Higaki\_Masakazu\_  \\WGNE\_BB2014\_TEPSupgrade\_higaki.pdf



\bibitem{Ito}   {\sc  ITO, K. } and  {\sc WU, C.   }, 2013. \emph{Typhoon-position-oriented sensitivity analysis. part I: theory and verification. } \emph{J. Atmos. Sci.}, {\bf 70}, 2525-2546.

    \bibitem{JMA} {\sc JAPAN METEORLOLOGICAL AGENCY}, 2014: \emph{Annual Report of the RSMC Tokyo-Typhoon Center 2013}. http://www.jma.go.jp/jma/ jma- eng/jma-center/rsmc-hp-pub-eg/AnnualReport/2013/Text/Text2013.pdf

\bibitem{Neumann} {\sc NEUMANN, C. J.}, 1972. \emph{An alternate to the HURRAN tropical cyclone forecast system.} \emph{NOAA Tech. Memo.} NWS SR-62, 22 pp.

\end{thebibliography}

%
\end{document}